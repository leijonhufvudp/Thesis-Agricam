%\noindent A combination of image processing techniques were identified to pre-process images of agar plates containing bacteria cultures. 
\noindent Mastitis is a common disease among cows in dairy farms. Diagnosis of the infection is today done manually, by analyzing bacteria growth on agar plates. However, classifiers are being developed for automated diagnostics using images of agar plates. Input images need to be of reasonable quality and consistent in terms of scale, positioning, perspective, and rotation for accurate classification. Therefore, this thesis investigates if a combination of image processing techniques can be used to match each input image to a pre-defined reference model. A method was proposed to identify important key points needed to register the input image to the reference model. The key points were defined by identifying the agar plate, its compartments, and its rotation within the image. \\

\noindent The results showed that image registration with the correct key points was sufficient enough to match images of agar plates to a reference model despite any varieties in scale, position, perspective, or rotation. However, the accuracy depended on the identification of the salient features of the agar plate.  
Ultimately, the work proposes an approach using image registration to transform images of agar plates based on a pre-defined reference model, rather than a reference image. 

